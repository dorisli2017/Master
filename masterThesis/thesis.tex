% Template for bachelor's/master's thesis for students at Karlsruhe Institute of Technology.
% Best for students majored in Computer Science or Mathematics.

% English version by Wei Zhou (ftfish@gmail.com)
% Based on the German version of Timo Bingmann


% This is just a template. Since there's no strict rule on how a bachelor's thesis must look like, every part of this template can be adapted.

\documentclass[12pt,a4paper,twoside]{scrartcl}

% Loading babel to enable automatic hypentation and multiple languages in the document.
% The last language in the option list will be used as default.
\usepackage[ngerman,english]{babel}
% Using T1 font encoding and the Latin Modern font
\usepackage[T1]{fontenc}
\usepackage{lmodern}
\usepackage{lscape}
\usepackage{afterpage}

% Using utf8 as file encoding.
\usepackage[utf8]{inputenc}
\usepackage[capbesideposition=outside,capbesidesep=quad]{floatrow}


% Page size. Using almost the whole A4 paper.
\usepackage[tmargin=22mm,bmargin=22mm,lmargin=20mm,rmargin=20mm]{geometry}


% Some standard packages for writing papers
\usepackage{latexsym,amsmath,amssymb,mathtools,textcomp}

% Paragraphs are not indented but there will be some space between paragraphs
\usepackage{parskip}
% For defining theorem-style environments like Lemma/Proof/Definition
\usepackage{amsthm}
% Fixing spacing problems before theorems due to \usepackage{parskip}
\begingroup
    \makeatletter
    \@for\theoremstyle:=definition,remark,plain\do{%
        \expandafter\g@addto@macro\csname th@\theoremstyle\endcsname{%
            \addtolength\thm@preskip\parskip
            }%
        }
\endgroup

% Some theorem-style environments
\newtheorem{theorem}{Theorem}[section]
\newtheorem{definition}[theorem]{Definition}
\newtheorem{lemma}[theorem]{Lemma}

% Setting equation numbers to <chapter>.<section>.<index>
\numberwithin{equation}{section}

% For inserting graphics into the document
\usepackage{graphicx}
\graphicspath{{images/}}

% For tables
\usepackage{array,multirow}

% Control layout of itemize, enumerate, description
\usepackage{enumitem}

\setlist[enumerate]{topsep=0pt}
\setlist[itemize]{topsep=0pt}
\setlist[description]{font=\normalfont,topsep=0pt}

\setlist[enumerate,1]{label=(\roman*)}

% TikZ for graphics in LaTeX
\usepackage{tikz}
\usetikzlibrary{calc}

% Have the current section and subsection in the header.
\usepackage{fancyhdr}
\usepackage{ltxtable} 
\fancypagestyle{plain}{
  \setlength\footskip{32pt}
  \fancyhead{}
  \fancyfoot{}
  \fancyfoot[LE,RO]{\normalsize\thepage}
  \renewcommand{\headrulewidth}{0pt}
  \renewcommand{\footrulewidth}{0pt}
}

\fancypagestyle{normal}{
  \setlength{\headheight}{20pt}
  \setlength\footskip{32pt}
  \fancyhead{}
  \fancyhead[LE]{\normalsize\textsc{\nouppercase{\leftmark}}}
  \fancyhead[RO]{\normalsize\textsc{\nouppercase{\rightmark}}}
  \fancyfoot{}
  \fancyfoot[LE,RO]{\normalsize\thepage}
  \renewcommand{\headrulewidth}{0.4pt}
  \renewcommand{\footrulewidth}{0pt}
}

% Hyperref for hyperlinks und cross refs
\usepackage{color}
\usepackage{lscape}
\usepackage[pagebackref]{hyperref}
\usepackage[all]{hypcap}
\usepackage{pbox}
\DeclareOldFontCommand{\bf}{\normalfont\bfseries}{\mathbf}
\hypersetup{
  pdftitle={SAT Solving with distributed local search},
  pdfauthor={Guangping Li}, 
  pdfsubject={Tabucol, \emph{\textbf{graph coloring problem}},parallel cooperative tabu search}, 
  colorlinks=true,
  pdfborder={0 0 0},
  bookmarksopen=true,
  bookmarksopenlevel=1,
  bookmarksnumbered=true,
  linkcolor=black,
  %linkcolor=black,
  citecolor=black,
  urlcolor=black,
  filecolor=black,
  pdfpagemode=UseNone,
  unicode=true,
}

% Add the word "page" for pagebackref's in the bibliography.
\renewcommand*{\backreflastsep}{, }
\renewcommand*{\backreftwosep}{, }
\renewcommand*{\backref}[1]{}
\renewcommand*{\backrefalt}[4]{%
  \ifcase #1 %
No citations.% use \relax if you do not want the "No citations" message 
  \or
(Page #2).%
  \else
(Pages #2).%
  \fi%
}


% For importing graphics from subdirectories.
\usepackage{import}

% Referencing figures, etc.
\newcommand{\reflst}[1]{\hyperref[#1]{Listing~\ref*{#1}}}
\newcommand{\refthm}[1]{\hyperref[#1]{Theorem~\ref*{#1}}}
\newcommand{\refdef}[1]{\hyperref[#1]{Definition~\ref*{#1}}}




% Package for inserting pseudo codes in the document.
\usepackage[ruled,vlined,linesnumbered,norelsize]{algorithm2e}
\DontPrintSemicolon
\def\NlSty#1{\textnormal{\fontsize{8}{10}\selectfont{}#1}}
\SetKwSty{texttt}
\SetCommentSty{emph}
\def\listalgorithmcfname{List of Algorithms}
\def\algorithmautorefname{Algorithm}
\let\chapter=\section % resolve a problem with algorithm2e

\begin{document}

%%%%%%%%%%%%%%%%%%%%%%%%%%%%%%%%%%%%%%%%%%%%%%%%%%%%%%%%%%%%%%%%%%%%%%
\pagestyle{empty} % no page number
\pagenumbering{alph}

% title page
\begin{titlepage}

  \begin{center}\large

    {\flushleft\includegraphics[height=17mm]{kit_logo_en.pdf} \hfill}
%    \includegraphics[height=20mm]{grouplogo-algo-blue.pdf}\quad\null

    \vfill

    \vspace*{2cm}

    {\bf\huge SAT Solving  \\ with distributed local search \par} 
    % Be sure to fill in the field pdftitle={} above
    % mit \par am Ende stimmt der Zeilenabstand
  

    \vfill
Master Thesis of \\

    \vspace*{15mm}
    {\bf Guangping Li} 

    \vspace*{15mm}

    At the Department of Informatics\\
Institute of Theoretical informatics, Algorithmics II 

    \vspace*{45mm}

    \begin{tabular}{rl}
      Advisors: & Dr. Tom{\' a}{\v s} Balyo \\
      & Prof. Dr. Peter Sanders  \\
    \end{tabular}
    
    \vspace*{10mm}

	% Deutsch
%    Institut für Theoretische Informatik, Algorithmik \\
%    Fakultät für Informatik \\
%    Karlsruher Institut für Technologie

    % English:
%     Institute of Theoretical Informatics, Algorithmics \\

    \vspace*{12mm}
  \end{center}
\afterpage{\null\newpage}
\end{titlepage}
\afterpage{\null\newpage}
%%%%%%%%%%%%%%%%%%%%%%%%%%%%%%%%%%%%%%%%%%%%%%%%%%%%%%%%%%%%%%%%%%%%%%
\vspace*{0pt}\vfill

\selectlanguage{ngerman}
\hrule\medskip

Hiermit versichere ich, dass ich diese Arbeit selbständig verfasst und keine anderen, als die angegebenen Quellen und Hilfsmittel benutzt, die wörtlich oder inhaltlich übernommenen Stellen als solche kenntlich gemacht und die Satzung des Karlsruher Instituts für Technologie zur Sicherung guter wissenschaftlicher Praxis in der jeweils gültigen Fassung beachtet habe.

\bigskip

\noindent
Karlsruhe, 1th September 2018 

% Hand-written signature!! %TODO

\vspace*{5cm}

\clearpage

%%%%%%%%%%%%%%%%%%%%%%%%%%%%%%%%%%%%%%%%%%%%%%%%%%%%%%%%%%%%%%%%%%%%%%

\vspace*{0pt}\vfill
\selectlanguage{english}
\begin{abstract}
\centerline{\bf Abstract}
{\noindent }
%Abstct %TODO
Stochastic local search (SLS) is an elementary technique for solving combinational problems. Probsat is an algorithm paradigm of the simplest SLS solvers for Boolean Satisfiability Problem (SAT), in which the decisions only based on the probability distribution. In the first section of this paper, we introduce an efficient Probsat heuristic. We experimentally evaluate and analyze the performance
of our solver in a combination of different techniques, including simulated annealing and tabu search. With the approach of formula partition, we introduce a parallel version of our solver in the second section. The parallelism improves the Efficiency of the solver. Using different random generator and other parameter settings in solving
the sub-formula can bring further improvement in performance to our parallel solver. 

\end{abstract}
\vfill
\afterpage{\null\newpage}
\selectlanguage{ngerman}
\begin{abstract}
\centerline{\bf Zusammenfassung}
{\noindent }
%german abstract %TODO 
Stochastische lokale Suche (SLS) stellt eine elementare Technik zur Lösung von komplizierten kombinatorischen Problemen dar. Probsat ist einer der einfachsten SLS-Solver für das Erfüllbarkeitsproblem der Aussagenlogik (SAT), bei dem die Entscheidungen nur auf der Wahrscheinlichkeitsverteilung basieren. Im ersten Teil dieser Arbeit stellen wir eine effiziente Probsat-basierte Heuristik vor.  Die Leistung unseres Algorithmus in einer Kombination verschiedener Techniken, einschließlich simulierter Abkühlung und Tabu Search wurde auch experimentell bewertet und analysiert. Mit dem Ansatz der Formelpartition wird im zweiten Teil eine parallele Version unseres Algorithmus eingeführt, die die Effizienz des Lösers verbessert. Die flexible Parametereinstellungen bei der Lösung der Teil-formeln kann eine weitere Verbesserung unseres Algorithmus bringen. 
\end{abstract}


\vfill\vfill\vfill
\clearpage

%%%%%%%%%%%%%%%%%%%%%%%%%%%%%%%%%%%%%%%%%%%%%%%%%%%%%%%%%%%%%%%%%%%%%%

\selectlanguage{english}
\pagestyle{plain}
\pagenumbering{roman}
  
% markiere sections im Seitenkopf links und subsections rechts
\renewcommand\sectionmark[1]{\markboth{\thesection\quad\MakeUppercase{#1}}{\thesection\quad\MakeUppercase{#1}}}
\renewcommand\subsectionmark[1]{\markright{\thesubsection\quad\MakeUppercase{#1}}}


\tableofcontents
\afterpage{\null\newpage}
\clearpage

%%%%%%%%%%%%%%%%%%%%%%%%%%%%%%%%%%%%%%%%%%%%%%%%%%%%%%%%%%%%%%%%%%%%%%
%%%%%%%%%%%%%%%%%%%%%%%%%%%%%%%%%%%%%%%%%%%%%%%%%%%%%%%%%%%%%%%%%%%%%%
\pagestyle{normal}
\pagenumbering{arabic}

\section{Introduction} 
\subsection{Problem/Motivation} 
The propositional satisfiability problem (SAT) is the first proven NP-complete problem. The problem is to determine whether an assignment of Boolean values to variables in a  Boolean formula (CNF) such that the expression evaluates to true. 
As one of the most studied problems in computer science, the SAT problem has many applications. Hard combinational problems can be resolved with appropriate Encoding as a sat problem.
The SAT problem has many applications in computer science like hard model checking, software verification or in automated planning and scheduling in artificial intelligence. 
Formula partition is one of the promising approaches in DPLL-like solvers. By giving the order to the variables according to a good formula partition, the search gets a relatively balanced decision tree. But formula partition is rarely used in a local search for the SAT problem. How to combine the formula partition with local search, will the local search benefit from the partitioning, if the formula partitioning can guide a parallel local search, are still open questions. 

\subsection{Content} 
The SAT problem, as a well-known NP-complete problem, has received a great deal of attention and different local search heuristics have been developed. This paper is a survey on the stochastic local search on SAT problem with a guide of formula partition. The following section summarizes the formal concept and introduces techniques used in this paper. 
 One class of the most straightforward but efficient stochastic local search algorithms Probsat is the algorithm basic in our paper. Probsat was proposed in 2012 by Adrian Balint and Uwe Schoening. Section 2 describes our Probsat algorithm and discusses our attempts to improve the original algorithm. By experimentally evaluation and comparison, some techniques turned out to be more efficient than the simple Probsat search.
With the partition of variables and its corresponding formulas, the problem can be separated into two subproblems of similar size.  In section 3, we search the potential benefit of formula partition in a parallel search. 
Section 4 describes the details in experiments and several empiric results mentioned in section 2 and section 3.  Section 5 concludes the paper with further works. 

\subsection{Definitions and Notations} 
The SAT Problem
The propositional satisfiability problem (SAT)  
A variable with two possible logical values TRUE or False is a Boolean variable. 
A literal is an atomic formula in propositional logic. Literals can either be a positive literal (a variable) or a negative literal (negation of a variable)
A clause is a disjunction(OR)of literal.
A formula is in conjunctive normal form (CNF) if it is a conjunction (AND) of clauses.  
An assignment assigns the truth value to each Boolean variable x in the formula. We say the assignment is satisfying a formula if the truth value of the formula with this assignment turns out to be true. Specifically, an assignment satisfies a clause, if one literal in the clause with value True in this assignment. A CNF formula is satisfying if the assignment satisfies all its clauses. We say an assignment a satisfying assignment if it satisfies the formula. Otherwise, we say there are conflicts in some clauses with this assignment, or some clauses are unsatisfying clauses with this assignment. 
The SAT problem is to determine whether a satisfying assignment exists for the given formula. If so, we denote the formula a satisfiable formula. 
A set is a container of unique elements. A set of 3 objects a, b, c is written as {a, b, c}. The
size of a set is the number of elements in the set.
Local search
For instance I of hard combinational Problem, there is a set of solutions.  According to the constraints of the problem, an object function (score) is used to evaluate the candidate solutions. The Goal of the local search is to find the solution of minimum cost (the solution with the maximal score).
A local search starts with an initial complete solution. According to some heuristic, the local search makes local changes to its current solution iteratively, hence the name local search. Starts from an initial solution, the search will evaluate the solutions which can be reached by applying a local change to the current solution and choose one of the neighbor solutions with local optimization. The search applies local moves until the optimal solution is reached, or in some cases, a generally good solution is reached.  Local search is widely used in hard combinational problems such as the traveling salesman problem [13] and the graph coloring problem [14]. 
In the Boolean satisfiability problem, a local search operates primarily as follows: The search start from a randomly generated assignment as the initial solution. If this current assignment satisfied the formula, the search stops with success. Otherwise, a variable is chosen depends on some criterion, which we further call pickVal. By change the assignment of the selected variable, a neighbor assignment of our current solution is reached in next step, which is also called variable flipping. A local search will move in the space of the assignments by making the variable flipping until a satisfying assignment is reached by the search.  
The heuristic used for the pickVal selection is based on some scores of the variables in the current assignment.Consider the assignment reached by taking a flip of the variable in the current assignment. The number of clauses satisfied in B,but not in A is called the breakcout of the local move from to. Accordingly, the number of clauses, which become satisfying because of the flipping, is the makecount. The number of newly satisfying clauses (make) minus the number of newly unsatisfying clauses (break), which is denoted as diffscore, represents the local improvement of the corresponding flipping. Apart from this, other aspects like the repetition number of one flip or the number of occurrences of the variables can be considered in a selection heuristic.
An example is the unit propagation embedded local solver EagleUp, which prefers flipping of variable with the highest number of occurrences in a formula to creates new unit clauses sooner. To get local improvement effectively, man can only consider variables in unsat clauses for the flipping selection. This process is called a focused local search and commonly used. 
The initial hope of the local search is that through iterative greedy local improvement the optimal global solution can be found.  The typical problem of the local search is that the greedy local improvement searches be trapped in local unattractive local optimal solution.  To avoid this, a worse solution then the current solution will be chosen for the next step (uphill moves). There are some techniques following used in local search with occasional uphill moves.

Stochastic local search
The stochastic local search will use the probability distribution of the scores of candidate solutions instead of the static decision. For the candidate moves, the probability of being chosen p(T(s)) corresponds to the scores of the solution. In this way, the advantage a move is, the probability of choosing it as the next step is higher.   This randomization will avoid the stuck of the search in a local minimum and decrease the misguiding of the heuristic in specific situations. 

Tabu local search 
 Tabu search is created by Fred W. Glover in 1986 [15] and formalized in 1989.  For recognize the loop in a suboptimal region, the search trace is recorded in the process by mark the recently reached neighboring assignments as tabu. The tabu moves will not be touched in the further search to discourage getting stuck in a region. By 

Simulated Annealing
Simulated Annealing is an approach of SLS solver to difficult combinational optimization problems proposed by Kirkpatrick, Gelatt, and Vecchi. This approach is inspired by the metallic process annealing of shaping the material by heating and then slowly cooling the material. This approach works as a local optimization algorithm guided by a controlling parameter temperature. By high temperature, an uphill move is allowed with high probability while only small steps are allowed in low temperature. The temperature is varying according to the score of the current situation.  For a current solution with a nearly optimal score, the temperature is near zero. For an unattractive local extreme with a poor score, the active search is tending to make uphill moves in high temperature.


The Probsat
Probsat is a class of SLS sat solver, which was introduced in 2012 by Adrian Balint and Uwe Schoening. In a probsat solver, the score of a candidate flip is solely based on the make and break score. The paradigm is as follows: At first, a completely random assignment is set as the initial assignment. In the initial solution, the truth value of The algorithm performs local moves by flip a variable in a random chosen unsatisfying clause and stops as soon as there are no unsatisfying clauses exists, which means the satisfying assignment is found. The flip probability p of the variables in the chosen clause is calculated in a function p(x,s) based on make and break score of the variable in the current assignment. The idea behind the probability function is to give the advantageous flipping relative high probability, but the other flipping has small chance to be chosen.  There are two kinds of functions are considered in the paper of Adrian Baliant: exponential function and a polynomial function. 
 An exponential function f is an exponential 2-parameter (cb and cm) function:
Or a function with polynomial decay, which is called a polynomial function in the original paper: 

As mentioned in the probsat paper, it turns out in experiments that the influence or make is rather weak, so the one parameter function like the following can lead to an efficient algorithm.
The pseudo code of a typical Probsat is shown below:
Algorithm 1: ProbSAT
Input: Formula F, maxTries, maxFlips
Output: satisfying assignment a or UNKNOWN
For I =1 to  maxTries do:
 a<- randomly generated assignment
 For I =1 to maxFlips do:
If(a satisfies  F) then 
Return a
C 
\subsection{The algorithms for comparison} %TODO
\section{Our local Solver}
\label{sec:Solving GCP by Tabucol}
\subsection{Probsat}
\subsection{Improvement through randomly generated solution}
\label{subsec:Improvement through Bias initialization}
\label{double choose (greedy and sls)}
\label{subsec:Improvement through Tabu}
\subsection{Improvement through random generator}
\subsection{Improvement through statistic}
\subsection{try star version}

\section{Our Parallel Algorithm}
\label{sec:Our parallel Algorithm}
\subsection{1st Approach: The pure portfolio approach (no partition)}
\subsection{2nd Approach: Star}
\subsection{3nd Approach: Hope}
\subsection{4th Approach: future}
\section{Evaluation} 
 %TODO some server information
\subsection{format}
\subsection{Benchmarks}
\label{benchmark}
\subsection{Used plots and tables}
Different plots and tables are used to illustrate the results of the following experiments.\\
\emph{\textbf{Comparison Table}}\\
\emph{\textbf{Scatter Plot}}\\
\emph{\textbf{Cactus Plot}}\\
\emph{\textbf{Advantage Plot}}\\
 \subsection{Automatic parameter optimization}
\subsection{Experiments}
\section{Conclusion}
\subsection{Further work}
\section{Bibliography}
\bibliographystyle{ieeetr}
\bibliography{references}
\end{document}
